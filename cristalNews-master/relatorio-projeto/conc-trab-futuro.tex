\chapter{Conclusões e Trabalho Futuro}
\label{chap:conc-trab-futuro}


\section{Introdução}
Neste capítulo são apresentadas as principais conclusões sobre o trabalho desenvolvido, bem como o trabalho futuro que pode ser desenvolvido.


\section{Conclusões Principais}
\label{sec:conc-princ}
Pode-se concluir que existem métodos eficazes que permitem retirar a informação presente na \emph{web}, estes seguem regras que se mostraram eficazes.\par 
Através de experiências realizadas e implementadas, como apresentadas em \emph{getparagraph} (\ref{chap4:sec:paragraph}), provou ser um método eficaz, pois em grande parte dos \emph{websites} utilizados, foram poucos os casos em que não devolveu um parágrafo válido para a notícia.\par
Este projeto provou também que existe uma forma diferente de ver notícias, pois podem ser observadas de uma forma resumida e onde se destacam conteúdos que o utilizador considera relevantes.
 



\section{Trabalho Futuro}
\label{sec:trab-futuro}
Grande parte do trabalho futuro passa pelo estudo e desenvolvimento de novas técnicas de aceder a \emph{websites} com conteúdo noticioso. A aplicação usa o recurso ao \emph{website}, \url{www.sapo.pt}, se a  estrutura do cógido \ac{HTML} for alterada, os métodos \emph{Jsoup} necessitam de ser alterados, tais como, as \emph{querys} ao documento \ac{HTML}.
Também passará por melhorar os métodos desenvolvidos, uma vez que alguns deles são muito pesados computacionalmente, idealmente era bom usar métodos de forma a poupar recursos e aumentando a velocidade no processamento da informação.


