\chapter{Introdução}
Este capítulo é dedicado à apresentação do tema do projecto bem como o porque da sua escolha, vão ser apresentados os seus objectivos e  estruturação deste documento.

\label{chap:intro}

\section{Enquadramento}
\label{sec:amb} % CADA SECÇÃO DEVE TER UM LABEL
% CADA FIGURA DEVE TER UM LABEL

Atualmente o jornalismo evolui no sentido em que cada vez é mais lógico e normal investir em notícias online, com isto todos os dias são colocadas enormes quantidades de notícias novas na Internet, ficando desta forma disponíveis para milhões de pessoas.\par
Devido a este grande fluxo de notícias só temos um perceção parcial da informação noticiosa que circula nos meios eletrónicos. Assim sendo um utilizador algumas tem muitas dificuldades em encontrar as notícias em que está interessado com a facilidade que desejaria. Ele só conseguirá encontrar noticias de maior importância providenciadas pelos organizações mais relevantes.



\section{Motivação}
\label{sec:mot}
Este projeto visa a criação de uma aplicação que a partir de uma grande e variedade de notícias produz uma vista clara para os tópicos escolhidos por um utilizador. Através de uma interface atualizada em tempo real e com o maior número de notícias possíveis, de acordo com o tópico escolhido, permite ao utilizador ter uma visão abrangente sobre os temas, reduzindo o tempo de pesquisa tornando mas simples a procura de notícias.

\section{Objetivos}
\label{sec:obj}
O objectivo deste projecto passa por testar, estudar e experimentar medidas de \emph{web srcaping} de forma a sintetizar o número de notícias presentes na \emph{web}. Através de uma aplicação o utilizador terá acesso a noticias separadas por tópicos, em cada tópico será possível realçar notícias consoante um léxico de palavras escolhido pelo utilizador. 

\section{Organização do Documento}
\label{sec:organ}

De modo a refletir o trabalho que foi feito, este documento encontra-se estruturado da seguinte forma:
\begin{enumerate}
\item O primeiro capítulo -- \textbf{Introdução} -- apresenta o projeto, a motivação para a sua escolha, o enquadramento para o mesmo, os seus objetivos e a respetiva organização do documento.
\item O segundo capítulo -- \textbf{Tecnologias Utilizadas} -- descreve os conceitos mais importantes no âmbito deste projeto, bem como as tecnologias utilizadas durante do desenvolvimento da aplicação.
\item O terceiro capítulo -- \textbf{Tecnologia e Ferramentas Utilizadas} -- descreve as tecnologias utilizadas durante o desenvolvimento do projecto
\item o quarto capítulo -- \textbf{Implementação e Testes} -- apresenta e discute os resultados obtidos através do uso da aplicação desenvolvida
\item O quinto capítulo -- \textbf{Conclusões e Trabalho Futuro} -- descreve as conclusões obtidas pelo autor, bem como algumas sugestões de desenvolvimento do projecto no futuro.
\end{enumerate}
